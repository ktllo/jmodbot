\documentclass{report}
\usepackage[utf8]{inputenc}
\usepackage[margin=0.675in]{geometry}
\usepackage[colorlinks = true]{hyperref}
\usepackage{minted}
\newcommand{\elem}[1]{\textless #1\textgreater}
\title{JModBot Configuration Guide\\Version 0.1}
\author{LO Kam Tao Leo\\\href{mailto:leolo@leolo.org}{leolo@leolo.org}}
\begin{document}
	\maketitle
	
	Copyright \textcopyright 2019, LO Kam Tao Leo
	
	All rights reserved.
	
	Redistribution and use in source and binary forms, with or without
	modification, are permitted provided that the following conditions are
	met:
	\begin{enumerate}
		\item  Redistributions of source code must retain the above copyright
	notice, this list of conditions and the following disclaimer. 
	
		\item Redistributions in binary form must reproduce the above copyright
	notice, this list of conditions and the following disclaimer in
	the documentation and/or other materials provided with the
	distribution.  
	
		\item The name of the author may not be used to
	endorse or promote products derived from this software without
	specific prior written permission.
	\end{enumerate}

	THIS SOFTWARE IS PROVIDED BY THE AUTHOR ``AS IS'' AND ANY EXPRESS OR
	IMPLIED WARRANTIES, INCLUDING, BUT NOT LIMITED TO, THE IMPLIED
	WARRANTIES OF MERCHANTABILITY AND FITNESS FOR A PARTICULAR PURPOSE ARE
	DISCLAIMED. IN NO EVENT SHALL THE AUTHOR BE LIABLE FOR ANY DIRECT,
	INDIRECT, INCIDENTAL, SPECIAL, EXEMPLARY, OR CONSEQUENTIAL DAMAGES
	(INCLUDING, BUT NOT LIMITED TO, PROCUREMENT OF SUBSTITUTE GOODS OR
	SERVICES; LOSS OF USE, DATA, OR PROFITS; OR BUSINESS INTERRUPTION)
	HOWEVER CAUSED AND ON ANY THEORY OF LIABILITY, WHETHER IN CONTRACT,
	STRICT LIABILITY, OR TORT (INCLUDING NEGLIGENCE OR OTHERWISE) ARISING
	IN ANY WAY OUT OF THE USE OF THIS SOFTWARE, EVEN IF ADVISED OF THE
	POSSIBILITY OF SUCH DAMAGE. 
	
	\vspace{0.5in}
	\noindent\makebox[\linewidth]{\rule{\textwidth}{0.4pt}}
	\vspace{0.5in}
	
	The key words ``MUST'', ``MUST NOT'', ``REQUIRED'', ``SHALL'', ``SHALL
	NOT'', ``SHOULD'', ``SHOULD NOT'', ``RECOMMENDED'',  ``MAY'', and
	``OPTIONAL'' in this document are to be interpreted as
      described in BCP 14 [RFC2119] [RFC8174] when, and only when, they
      appear in all capitals, as shown here.
      %in this document are to be interpreted as described in
	%RFC 2119\footnote{\url{https://tools.ietf.org/html/rfc2119}}.
	\tableofcontents
	\chapter{Overview}
	The configuration file for JModBot is an XML file. The root element is ``configuration''. 
	
	\chapter{Network}
	\chapter{Database}
	\section{Custom Hibernate Configuration}
	You MAY add the following Hibernate properties by using Generic Configuration Parameter (See Chapter~\ref{c:gcp}).
	The custom hibernate configuration MUST be in bot scope.
	\chapter{Identity}
	\chapter{Modules}
	\section{Static vs Dynamic module}
	There are two different kind of module for JModBot, static and dynamic. These two kind of modules have a lot of differences for it's loading mechanism. In general, a module (except CAP modules, which are always static and handled differently) should be able to be loaded statically or dynamically.
	\section{\elem{staticModule} and \elem{dynamicModule}}\label{key_staticModule}\label{key_dynamicModule}
	The XML element \elem{staticModule} and \elem{dynamicModule} are used to indicate a static module or a dynamic module should be loaded respectively.
	\chapter{Module specific configuration}
	\chapter{Generic Configuration Parameters}\label{c:gcp}
	Some of the module, or function, may requires, or permit some extra configuration to be provided. In order
	to provide enough flexibility, you can includes \elem{custom-configs} element to various
	different configuration scope.
	\section{Configuration Scope}
	\section{\elem{custom-configs}}
	Each \elem{custom-configs} MUST contains one or more element 
	\elem{custom-config}. 
    \section{\elem{custom-config}}
    Each \elem{custom-config} element MUST contains two attributes, ``name'' and 
    ``value''. The ``name'' will be acting as the key to the configuration value.
    
    The scope of the custom configuration will be depends on its location.
	\appendix
	\chapter{XML Schema for JModBot Configuration}
	\inputminted[linenos]{xml}{config.xsd}
	\chapter{Sample Configuration}
	\inputminted[linenos]{xml}{bot.sample.xml}
\end{document}